\documentclass[man]{apa7}
\usepackage[english]{babel}
\usepackage{graphicx}
\usepackage{array}
\usepackage{float}
\usepackage{longtable}
\pagestyle{plain}
\shorttitle{}

% Portada
\title{Workshop \#1}
\author{Sergio Andres Diaz Cuervo - 20251020166  \\ Johan Esmit Sichacá González - 20242020313 \\ john Mario jimenez Becerra - 20251020047}
\affiliation{Professor: Carlos Andres Sierra Virguez \\[0.5cm] Subject: OOP}
\date{\today}

\begin{document}
\maketitle

\section{Functional Requirements}

\subsection{User Management (The program must:)}
\begin{itemize}
    \item Be able to register new users.
    \item Be able to log in registered users.
    \item Allow modification of user data.
\end{itemize}

\subsection{Activity and Event Registration (The program must:)}
\begin{itemize}
    \item Allow the user to create an event.
    \item Allow the user to set: event date, name, and a brief description.
    \item Display the created event in their calendar.
    \item Allow the user to create a task list.
    \item Allow the user to set: task name, estimated completion date, subject or topic of the task, and its status (completed or not).
    \item Display pending tasks as “Tasks” in their calendar.
\end{itemize}

\subsection{Event Planning}
Upon registration, the user has access to the creation of 2 calendars:
\begin{itemize}
    \item The first titled “Personal” (only contains their created events).
    \item The second shows only the tasks they created, but in the calendar, as “pending.”
\end{itemize}

\subsection{Database Connectivity}
\begin{itemize}
    \item The user can have more than one task list and can choose which one to import into their calendar (only one can be viewed at a time).
    \item When the user registers an event in the calendar, it must also be viewable outside of it, in the events section.
\end{itemize}

\subsection{Data Sharing}
\begin{itemize}
    \item The user can share their calendars with other people.
    \item When sharing a calendar, the receiving users can only view it; they cannot modify it.
\end{itemize}

\section{Non-Functional Requirements}

\subsection{Performance}
\begin{itemize}
    \item The application must support at least 5 users simultaneously.
    \item Event creation must take less than 5 seconds.
    \item Task list creation must take less than 5 seconds.
    \item Importing a database must not take more than 10 seconds.
\end{itemize}

\subsection{Security}
\begin{itemize}
    \item All user passwords must remain confidential.
    \item Usernames will only be visible when the user decides to share one of their calendars.
\end{itemize}

\subsection{Availability}
The software must be available 99\% of the time; that is, it should remain online at all times.

\subsection{Usability}
\begin{itemize}
    \item The system interface must be intuitive and allow a new user to design their task list or calendar easily.
    \item Calendars must be in a simple format, without complicating the reading or understanding of dates or data.
    \item Lists must be as dynamic as possible, avoiding excessive information and visual overload.
    \item The color scheme of the interface must be optimal to prevent user eye strain.
\end{itemize}

\subsection{Maintainability}
\begin{itemize}
    \item The code must be documented according to the organization’s standards.
    \item The system must be designed in a modular way to facilitate future updates.
\end{itemize}

\section{User Stories}
\begin{longtable}{|p{3cm}|p{2cm}|p{5cm}|p{5cm}|}
\hline
\textbf{Title} & \textbf{Priority} & \textbf{User Story} & \textbf{Acceptance Criteria} \\
\hline
User Register & High & As a new user I want to register in the application so that I can start making my activities. & Given I enter for the first time, when I fill in the form and submit it, then my account is created. \\
\hline
Login & High & As a user I want to log in with my credentials so that I can access my calendars and tasks. & Given I already have an account, when I enter a valid email and password, then I access the application. \\
\hline
Event Creation & High & As a user I want to create an event with date, name, and description so that I can organize my activities. & Given I am in my calendar, when I click “create event” and fill in the data, then the event appears on the selected date. \\
\hline
Task List within Event & High & As a user I want to create a task list within an event so that I can track specific subtasks. & Given I already have an event created, when I add tasks and statuses (pending/done), then I see the progress inside the event. \\
\hline
Pending Tasks View & High & As a user I want to see all my pending tasks in a single list so that I know what I still need to do. & Given I have multiple events with tasks, when I open the “pending” view, then the list appears organized by date. \\
\hline
Event Modification & High & As a user I want to edit the name or date of an event so that I can fix it without creating a new one. & Given I have an existing event, when I change its data and save, then the event is updated. \\
\hline
Share Calendar & Low & As a user I want to share my calendar with other users so that they can view my activities. & Given I already have events created, when I select “share” and enter an email, then the other user gets read-only access. \\
\hline
\end{longtable}

\section{CRC Cards}
\begin{longtable}{|p{3cm}|p{7cm}|p{6cm}|}
\hline
\textbf{Class} & \textbf{Responsibilities} & \textbf{Collaborators} \\
\hline
User & Log in, Sign in; Assign tasks; Store profile data; Manage preferences & Task, Notification, Progress statistic, Check list, Register \\
\hline
Notification & Notify the user; Inform about the expiration of an event & User, Task, Calendar \\
\hline
Progress Statistic & Create daily, weekly and monthly statistics; Show tasks missing progress; Show task percentage completed & User, Task, Check list, Calendar, Notification \\
\hline
Check List & Show the status of a task's completion; Show the task’s specific data (date, category, name, etc.); Mark tasks as completed & User, Task, Progress Statistic \\
\hline
Calendar & Show tasks/events general data in daily, weekly, and monthly view; Synchronize with time zone; Update in real time & User, Task, Check list, Notification, Progress statistic \\
\hline
Tasks & Save all the information about a task; Allow to edit or eliminate a task; Define recurrence if repetitive & User, Task, Notifications, Progress statistics \\
\hline
Register & Receive and validate data from registration form; Create the User entity in database & User, Notification \\
\hline
\end{longtable}

\section{Mockups}

\begin{figure}[H]
\centering
\includegraphics[width=0.8\linewidth]{1.jpg}
\caption{Mockup 1}

\vspace{1cm}

\includegraphics[width=0.8\linewidth]{2.jpg}
\caption{Mockup 2}
\end{figure}

\begin{figure}[H]
\centering
\includegraphics[width=0.8\linewidth]{3.jpg}
\caption{Mockup 3}

\vspace{1cm}

\includegraphics[width=0.8\linewidth]{4.jpg}
\caption{Mockup 4}
\end{figure}

\begin{figure}[H]
\centering
\includegraphics[width=0.8\linewidth]{5.jpg}
\caption{Mockup 5}

\vspace{1cm}

\includegraphics[width=0.8\linewidth]{6.jpg}
\caption{Mockup 6}
\end{figure}

\begin{figure}[H]
\centering
\includegraphics[width=0.8\linewidth]{7.jpg}
\caption{Mockup 7}

\vspace{1cm}

\includegraphics[width=0.8\linewidth]{8.jpg}
\caption{Mockup 8}
\end{figure}

\begin{figure}[H]
\centering
\includegraphics[width=0.8\linewidth]{9.jpg}
\caption{Mockup 9}

\vspace{1cm}

\includegraphics[width=0.8\linewidth]{10.jpg}
\caption{Mockup 10}
\end{figure}

\end{document}
